\begin{enumerate}
    \item Modelagem geométrica e o uso de malhas poligonais são uma parte intrínseca da computação gráfica. Com uma gama de aplicações e variados usos, a maior parte da humanidade já interagiu com malhas de alguma forma. Sendo assim, quais são alguns destes usos? Procure também encontrar uma malha que você possa manipular de alguma forma. 
    
    \item Construa a matriz de adjacência para uma pirâmide de base quadrada.
    
    \item Construa a matriz de adjacência do exercício anterior em Matlab/Octave.
    
    \item Gere um código em Matlab/Octave que receba uma matriz de adjacências e calcule o laplaciano topológico $L_s$.
    
    \item Gere um código em Matlab/Octave que calcule as $\delta$-coordenadas (com a ponderação padrão vista em \ref{eq_delta}), a partir das coordenadas cartesianas e da matriz de adjacências.
    
    \item Em softwares de visualização de malhas com interação do usuário, é fundamental que o tempo de processamento seja pequeno. Quais as vantagens das matrizes $L$, $L_s$ e $D$ vistas anteriormente em questão de eficiência computacional?
    
    \item Para a representação de formas, é necessário possuir informações de conectividade da malha, como a matriz de adjacências. Como é possível gerar a matriz de adjacência, caso se saiba que a malha é a representação de:
    
\begin{enumerate}
    \item Curva aberta contínua no $\mathbb{R}^2$, em que $v_i \approxeq v(t)$, e os índices estão corretamente ordenados;
    
    \item Curva fechada contínua no $\mathbb{R}^2$, em que $v_i \approxeq v(t)$, e os índices estão corretamente ordenados;
\end{enumerate}

    \item Na seção \ref{LB_matlab}, os pontos âncora foram selecionados como pontos igualmente espaçados. Sugira alguma outra forma para coletá-los, e analise brevemente sua eficiência de representação.

    \item Como visto na seção \ref{Pondera}, alguns dos métodos utilizados para ponderação das $\delta$-coordenadas incluem o uso de pesos cotangentes ou pesos relativos ao número de vizinhos de um vértice. Quais são alguns outros métodos que podem ser utilizados para a ponderação de $\delta$-coordenadas ponderadas?

\end{enumerate}

\subsection{Resoluções}
\begin{enumerate}
    \item De fato, existe um grande número de áreas que utilizam malhas poligonais, desde animação através de computação gráfica, aplicações para decoração de imóveis e arquitetura, modelos computacionais de jogos virtuais, reconhecimento de faces e inúmeras outras aplicações. Também é possível encontrar em diversos repositórios na internet malhas que podem ser carregadas em aplicações que permitem desde a movimentação de câmeras para observação da malha até manipulação direta como torcer, esticar e deformar a malha em geral.
    
    \item Uma possível matriz de adjacência para uma pirâmide em que os pontos 1, 2, 3 e 4 compõe a base é:

$$P = \begin{pmatrix}
0 & 1 & 1 & 0 & 1\\
1 & 0 & 0 & 1 & 1\\
1 & 0 & 0 & 1 & 1\\
0 & 1 & 1 & 0 & 1\\
1 & 1 & 1 & 1 & 0\\
\end{pmatrix}$$

Para transformar essa matriz de adjacência na matriz de uma malha triangular seria necessário adicionar uma aresta entre os pontos 2 e 3 ou 1 e 4.

    \item Vide códigos \textit{geraAdjacenciaPadrao.m} e \textit{geraAdjacenciaPiramide.m}.

    \item Vide código \textit{geraLaplaciano.m}.
    
    \item Vide códigos \textit{geraPesosPadrao.m} e \textit{geraDeltaCoords.m}.

    \item As matrizes descritas são esparsas, o que permite a utilização de algoritmos e estruturas de dados especificas. Além disso, também é possível aplicar algumas decomposições de resolução de sistemas linear (como a decomposição de \textit{Cholesky}) para obter uma maior eficiência.
    
    \item 
    \begin{enumerate}
        \item Caso a malha represente uma curva, e os índices dos nós estejam ordenados conforme o tempo, basta ligar cada vértice $i$ com o predecessor $i-1$ e sucessor $i+1$, para todo $i \in \{2, \dots, n-1\}$. Os extremos (primeiro e último vértices) devem ser tratados a parte: como a curva é aberta, o vértice $1$ está ligado apenas com o $2$, e o vértice $n$ está somente conectado com o $n-1$.
        
        \item Similarmente ao item anterior, basta ligar cada vértice $i$ com o predecessor $i-1$ e sucessor $i+1$, para todo $i \in \{2, \dots, n-1\}$. Porém, como a curva é fechada, além das conexões citadas anteriormente, os vértices $1$ e $n$ também são conectados.
    \end{enumerate}
    
    \item Existem várias sugestões, como pegar os vértices com maior e menor curvatura (podem dar uma melhor representação de curvas), pontos aleatórios (podem ser probabilisticamente melhores ou piores que igualmente espaçados) ou escolher iterativamente os vértices de maneira gulosa (de forma a minimizar a função de erro do método dos mínimos quadrados, que pode ser superior às outras formas discutidas anteriormente).
    
    \item Existem diversas possibilidades para a ponderação de $\delta$-coordenadas. A equação \ref{eq_forpon} descreve a ponderação através dos termos $\mathbf{c_i}$ e $\mathbf{w_{ij}}$, ou seja, alterando estes termos é possível utilizar uma infinidade de métodos de ponderação diferentes. Por exemplo, como mencionado na seção \ref{MD} é possível fazer com que o peso de cada aresta $\mathbf{w_{ij}}$ seja inversamente proporcional ao seu tamanho, isto é, a distância entre $\mathbf{v_i}$ e $\mathbf{v_j}$.
    
\end{enumerate}
