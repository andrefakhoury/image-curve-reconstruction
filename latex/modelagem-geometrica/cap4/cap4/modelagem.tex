%%%%%%%%% tirei essa parte, da pra explicar isso aqui junto com a "introducao" (que agora chama `Modelagem geometrica`)

Malhas geométricas, também denominadas como malhas poligonais, também possa ser descrita por polígonos de n-vért. Um dos modelos mais tradicionais de malhas geométricas são as malhas triangulares. É em cima deste modelo que as próximas seções serão desenvolvidas.

\begin{defi}[Malha Triangular]
Seja $V$ um conjunto de pontos em um espaço bidimensional ou tridimensional e $A$ o conjunto de arestas que conectam estes pontos. A malha triangular $M$ é o conjunto de triângulos descrito por M = (V,A), sendo que estes triângulos estão conectados por seus vértices e arestas em comum.
\end{defi}
Utilizando Malhas Dinâmicas



Escolhendo pontos para modificar a malha

Para escolher o ponto a ser modificado em uma região de interesse, é importante selecionar o ponto com o maior rank, ou seja, o vértice mais conectado a outros vértices desta dada região.

\begin{defi}[Rank]
\end{defi}
Seja $M$ uma Malha Triangular. Seja $k$ o número de componentes conectados de M e $n$ o número de vértices adjacentes a um vértice $v$.
O rank de v é dado por 
\begin{equation}
rank(v) = n - k
\end{equation}
Com um grau de liberdade, pois elementos conectados sempre estarão conectados a pelo menos outro vértice.


\begin{itemize}
    \item Como faz pra que tenha o formato que você queira (movimentando); Malhas Dinâmicas Igor/ Apresentação da Olga
    \item Como formar e resolver a matriz;
    \item Como escolher os pontos;
\end{itemize}