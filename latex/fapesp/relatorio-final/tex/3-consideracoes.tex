Durante todo o período referente ao projeto foram realizadas reuniões semanais com o grupo de professores e alunos do projeto temático em que esta iniciação científica está inserida. Nestas, foram preparados e apresentados alguns seminários, que serviram para que todos pudessem acompanhar o andamento do projeto de todos, dar \textit{feedbacks} e planejar os próximos passos a serem executados.

Outro tópico relacionado com o projeto (e que também foi apontado no relatório parcial) foi que, durante o segundo semestre de 2020, o professor Dr. Antônio Castelo Filho, pesquisador principal do projeto temático, lecionou a disciplina ``Modelagem Geométrica'' para alunos de graduação e com espelho em ``Tópicos em Análise Numérica II (Variedades Computacionais)'' para a pós-graduação. Nela, foram abordados diversos tópicos referentes a variedades computacionais. No caso particular deste projeto, a atividade diretamente relacionada referiu-se à representação de curvas em coordenadas diferenciais pelo método descrito em \citeonline{Sorkine2006}. Nesta disciplina, cada estudante participou ativamente no desenvolvimento de um capítulo de um livro, que ainda será publicado.

Além disso, também foram realizados dois \textit{workshops} relacionados ao projeto temático, em que o bolsista participou como ouvinte. O último tópico a ser mencionado é a participação deste projeto no Simpósio Internacional de Iniciação Científica e Tecnológica da USP (SIICUSP) de 2021, que será realizado em outubro de 2021.

Este relatório teve como objetivo listar as atividades realizadas no período de março a setembro de 2021 e exibir um panorama geral de desenvolvimento do projeto. Apesar de algumas dificuldades encontradas neste período (referentes à pandemia de COVID-19) e de alterações sobre o cronograma original, pode-se finalizar o projeto no tempo previsto e foi possível promover conhecimento entre os membros do grupo de pesquisa do projeto temático em que esta pesquisa está inserida.