De acordo com a organização inicial e levando em consideração as alterações discutidas na seção \ref{sec::reordenacao}, as próximas atividades a serem realizadas serão as seguintes:

\begin{enumerate}[noitemsep]
	\item \textbf{Extração dos pontos importantes}: estudar e implementar técnicas de extração das características robustas, a partir da análise de curvatura.
	\item \textbf{Combinação dos algoritmos}: juntar os códigos implementados de pré-processamento de imagens, extração dos pontos importantes e reconstrução de curvas em um único fluxo de desenvolvimento, para que eles se comuniquem e se facilitem os testes.
	\item \textbf{Avaliação e testes}: executar o programa implementado em instâncias do banco de imagens \citeonline{imageclef2011}, e inferir a qualidade das implementações feitas.
	\item \textbf{Desenvolvimento do relatório final}: escrita do relatório final.
\end{enumerate}

\noindent e seguirão o seguinte cronograma, visto na tabela \ref{tab:proxcronograma}:

\begin{table}[htb]
	\footnotesize
	\centering
	\vspace{0.5em}
	\setlength{\tabcolsep}{0.05in}
	\begin{tabular}{|c|c|c|c|c|c|c|c|}
		\hline
		Atividades
		& \multicolumn{7}{c|}{Meses} \\
		\cline{2-8}
		& Março & Abril & Maio & Junho & Julho & Agosto & Setembro \\ \hline
		Extração dos pontos importantes & $\bullet$ & $\bullet$ & $\bullet$ & & & & \\ \hline
		Combinação dos algoritmos & & & $\bullet$ & $\bullet$ & & & \\ \hline
		Avaliação e testes & & & & $\bullet$ & $\bullet$ & $\bullet$ & \\ \hline
		Desenvolvimento do relatório final & & & & & & $\bullet$ & $\bullet$ \\ \hline
	\end{tabular}
	\caption{Cronograma de atividades para os próximos meses de trabalho.}
	\label{tab:proxcronograma}
\end{table}