A modelagem geométrica, também denominada modelagem de superfícies, consiste de um conjunto de técnicas e algoritmos utilizados para modelar determinadas formas matemáticas, sujeitas a condições particulares de forma e suavidade.

Aplica conhecimentos de matemática aplicada e geometria computacional, e é amplamente utilizada em diversas áreas, como a engenharia automotiva e aeronáutica, em especial em CAD/CAM \textit{(computer aided design/manufacturing)}, por seu alto poder em modelagem de superfícies.

Em contextos tridimensionais, alguns autores utilizam o termo ``modelagem de sólidos'' \cite{agoston2005} - contudo, este termo não será empregado neste livro, tendo em vista que serão abordadas as modelagens de objetos geométricos quaisquer. 

Uma das possibilidades de se realizar a modelagem geométrica é com a utilização da discretização do operador de Laplace-Beltrami. Os principais métodos para esta discretização podem ser divididos em dois grupos principais: baseados em malha (\textit{mesh-based}) e baseados em ponto (\textit{point-based}) \cite{petronetto2013}. Neste capítulo, descreveremos uma abordagem baseada em malha, descrita em \cite{sorkine2006}.
